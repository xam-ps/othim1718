%-------------------------------------------------------------
% Packages
%-------------------------------------------------------------
\usepackage[
	automark,								 	% Kapitel in Kopfzeile
	headsepline,								% Trennlinie unter Kopfzeile
	ilines										% Trennlinie linksbündig ausrichten
]{scrpage2}


\renewcommand{\familydefault}{\sfdefault}
\usepackage[T1]{fontenc}
%\usepackage[ngerman]{babel}
%\usepackage[ansinew]{inputenc}
%\usepackage[T1]{fontenc}
\usepackage[latin1]{inputenc}
%\usepackage[utf8]{inputenc}
\usepackage{geometry}
\usepackage[acronym,toc,automake]{glossaries}
\geometry{verbose,tmargin=2.5cm,bmargin=3.5cm}
\pagestyle{plain}
\setlength{\parskip}{\medskipamount}
\setlength{\parindent}{0pt}
\usepackage{color}
\usepackage{babel}
\usepackage{array}
\usepackage{varioref}
\usepackage{textcomp}
\usepackage{multirow}
\usepackage{amsfonts}
\usepackage{amsthm}
\usepackage{amsmath}
\usepackage{graphicx}
\usepackage{setspace}
\usepackage{nomencl}
% the following is useful when we have the old nomencl.sty package
\providecommand{\printnomenclature}{\printglossary}
\providecommand{\makenomenclature}{\makeglossary}
\makenomenclature
\setstretch{1.2}

\makeatletter

%%%%%%%%%%%%%%%%%%%%%%%%%%%%%% LyX specific LaTeX commands.
%% Because html converters don't know tabularnewline
\providecommand{\tabularnewline}{\\}

%%%%%%%%%%%%%%%%%%%%%%%%%%%%%% User specified LaTeX commands.
% verschieden Symbole, Zeichen wie (c), ¤
\usepackage{textcomp,units}

% Mehr Platz zwischen Tabelle und Untertitel
\usepackage{caption}
\captionsetup[table]{skip=10pt}

%Kapitelzahl sehr groß
\makeatletter% siehe De-TeX-FAQ 
 \renewcommand*{\chapterformat}{% 
   \begingroup% damit \unitlength-Änderung lokal bleibt 
     \setlength{\unitlength}{1mm}% 
     \begin{picture}(10,10)(0,5) 
       \setlength{\fboxsep}{0pt} 
       %\put(0,0){\framebox(20,40){}}% 
       %\put(0,20){\makebox(20,20){\rule{20\unitlength}{20\unitlength}}}% 
       \put(10,15){\line(1,0){\dimexpr 
           \textwidth-20\unitlength\relax\@gobble}}% 
       \put(0,0){\makebox(10,20)[r]{% 
           \fontsize{28\unitlength}{28\unitlength}\selectfont\thechapter 
           \kern-.05em% Ziffer in der Zeichenzelle nach rechts schieben 
         }}% 
       \put(10,15){\makebox(\dimexpr 
           \textwidth-20\unitlength\relax\@gobble,\ht\strutbox\@gobble)[l]{% 
             \ \normalsize\color{black}\chapapp~\thechapter\autodot 
           }}% 
     \end{picture} % <-- Leerzeichen ist hier beabsichtigt! 
   \endgroup 
}

\usepackage{ %a4wide,
            ellipsis, fixltx2e, mparhack,   %Fehlerkorrektur für Marginalien
            booktabs, longtable             %schönere Tabellen
}  

%\usepackage[automark]{scrpage2}
%\automark[chapter]{chapter}
%\clearscrheadfoot
%\ohead{\\\headmark}
%\ihead{\includegraphics[scale=0.15]{logo.jpg}}%\pagemark}
%\ofoot[\pagemark]{\pagemark}


%Kurzfassung und Abstract (englisch) auf eine Seite
\renewenvironment{abstract}{
    \@beginparpenalty\@lowpenalty
      \begin{center}
        \normalfont\sectfont\nobreak\abstractname
        \@endparpenalty\@M
      \end{center}
}{
    \par
}



% schönerer Blocksatz!!
\usepackage{microtype}

\usepackage{ifpdf} % part of the hyperref bundle
\ifpdf % if pdflatex is used

%set fonts for nicer pdf view
 \IfFileExists{lmodern.sty}{\usepackage{lmodern}}
  {\usepackage[scaled=0.92]{helvet}
    \usepackage{mathptmx}
    \usepackage{courier}
     }
\fi

 % the pages of the TOC are numbered roman
 % and a pdf-bookmark for the TOC is added
 \pagenumbering{roman}
 \let\myTOC\tableofcontents
 \renewcommand\tableofcontents{
   %\pdfbookmark[1]{Contents}{}
   \myTOC
   \clearpage
   \pagenumbering{arabic}}

%Bezeichungen anpassen
%Babelpaket muß zuvor geladen werden
%\usepackage[english]{babel}
\addto\captionsngerman{ 
%\renewcommand{\figurename}{Abb.}% 
%\renewcommand{\tablename}{Tab.}% 
%\renewcommand{\abstractname}{Summary}
%\renewcommand{\nomname}{Abkürzungen}
}

% Alle Querverweise und URLs als Link darstellen
% In der PDF-Ausgabe
 \usepackage[colorlinks=true, bookmarks, bookmarksnumbered, bookmarksopen, bookmarksopenlevel=1,
  linkcolor=black, citecolor=black, urlcolor=blue, filecolor=blue,
  pdfpagelayout=OneColumn, pdfnewwindow=true,
  pdfstartview=XYZ, plainpages=false, pdfpagelabels,
  pdfauthor={LyX Team}, pdftex,
  pdftitle={LyX's Figure, Table, Floats, Notes, and Boxes manual},
  pdfsubject={LyX-documentation about figures, tables, floats, notes, and boxes},
  pdfkeywords={LyX, Tables, Figures, Floats, Boxes, Notes}]{hyperref}

%mehr Platz zwischen Überschrift und Tabelle
\newcommand{\@ldtable}{}
\let\@ldtable\table
\renewcommand{\table}{ %
                 \setlength{\@tempdima}{\abovecaptionskip} %
                 \setlength{\abovecaptionskip}{\belowcaptionskip} %
                 \setlength{\belowcaptionskip}{\@tempdima} %
                 \@ldtable}

%In dieser Arbeit wird auf die Nomenklatur als Abkürzungsverzeichnis verzichtet. Bei Wunsch wieder aktivieren.
%Nomenklatur als Abkürzungsverzeichnis verwenden
%\renewcommand{\nomname}{Abkürzungsverzeichnis}
%\renewcommand{\nomlabelwidth}{20mm}

%Nomenklatur als Glossar verwenden
%Nur Noetig wenn auch Glossar verwendet wird.
\renewcommand{\nomname}{Glossary}

%Farbe für Programmcode festlegen
\definecolor{lightgray}{rgb}{0.8,0.8,0.8}

\AtBeginDocument{
  \def\labelitemiii{\(\circ\)}
}

\makeatother

\usepackage{listings}
\addto\captionsamerican{\renewcommand{\lstlistingname}{Listing}}
\addto\captionsngerman{\renewcommand{\lstlistingname}{Listing}}
\renewcommand{\lstlistingname}{Listing}