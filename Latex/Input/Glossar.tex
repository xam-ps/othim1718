%-------------------------------------------------------------
% Abkürzungs- & Symbolverzeichnis
%-------------------------------------------------------------
% Abkürzungen

% A
\newacronym{ascii}{ASCII}{American Standard Code for Information Interchange}

\newacronym{KI}{KI}{Künstliche Intelligenz}
\newacronym{AI}{AI}{Artificial Intelligence}
\newacronym{TF}{TF}{TensorFlow}
\newacronym{NN}{NN}{Neuronale Netze}
\newacronym{TPU}{TPU}{Tensor Processing Unit}

\newglossaryentry{Accuracy}{
  name={Accuracy},
  description={Prozentualer Anteil der richtig vorhergesagten Ausgaben von allen Vorhersagen, gemessen an einer Teilmenge der Trainingsdaten}
}

\newglossaryentry{Cluster}
{
  name={Cluster},
  description={Eine Gruppierungen von Merkmalsträger deren Merkmale ein gemeinsames Muster aufweisen}
}

\newglossaryentry{Clustering}
{
  name={Clustering},
  description={Das Bilden von Clustern},
  see={Cluster}
}

\newglossaryentry{Feature}{
  name={Feature},
  plural= {Features},
  description={Ein Merkmal eines Merkmalträgers, welches als Eingabe für
  				lernende Algorithmen verwendet wird}
}

\newglossaryentry{Label}{
  name={Label},
  plural=Labels,
  description={Die von einem Lehrer festgelegte Ausgabe zu den Eingabewerten (Features) eines Merkmalsträger}
}

\newglossaryentry{ML}{
  type=\acronymtype,
  name={ML},
  description={Machine Learning, bzw. Maschinelles Lernen},
  first={Machine Learning (ML)}
}

\newglossaryentry{Model}
{
  name={Modell},
  plural=Modelle,
  description={Die Funktion, die mit Hilfe der Trainingsdaten gefunden wird und die für die Vorhersage von Ausgaben für neue Merkmalsträger genutzt werden kann}
}

\newglossaryentry{NN}{
  type=\acronymtype,
  name={Neuronales Netz},
  plural = {Neuronale Netze},
  description={Neuronales Netz, bzw. Neural Network},
  first={Neuronale Netze (NN)}
}

\newglossaryentry{SL}{
  name={Supervised Learning},
  description={Das Lernen, das einen Lehrer voraussetzt. Eine Funktion zur Vorhersage von Ausgaben wird über das Anpassen an gelabelte Testdaten erstellt}
}

\newglossaryentry{Testdaten}
{
  name={Testdaten},
  description={Die Menge aller Merkmalsträger, die beim Supervised Learning zum}
}

\newglossaryentry{Trainingsdaten}
{
  name={Trainingsdaten},
  description={Die Menge aller Merkmalsträger, die beim Supervised Learning zum Trainieren des Modells verwendet wird}
}

\newglossaryentry{UL}
{
  name={Unsupervised Learning},
  description={Das Lernen, das keinen Lehrer benötigt. Der Algorithmus versucht, Muster in den Daten zu finden, ohne die Bereitstellung von zusätzlichem Wissen über die Daten}
}