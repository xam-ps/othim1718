%-------------------------------------------------------------
% Abstract
%-------------------------------------------------------------


\newpage{}

~

\vspace{17.1mm}


\noindent \begin{flushleft}
\textbf{\huge{}Abstract}
\par\end{flushleft}{\huge \par}

Die hier vorliegende Arbeit befasst sich mit der Einführung in die plattformunabhängige Bibliothek \gls{TF}, welche seit Ende 2015 als Open-Source zur Verfügung steht und für den Bereich Maschinelles Lernen vom Google Brain Team entwickelt wurde. Das Ziel dieser Arbeit ist es, dem Leser den Einstieg in TensorFlow zu erleichtern und einen Überblick über die vorhandenen Möglichkeiten dieses umfangreichen Frameworks zu geben.

Zu Beginn wird kurz die Geschichte des Maschinelles Lernens vorgestellt. Hierbei wird auf einige wichtige Meilensteine eingegangen, sowie auf Bereiche in denen der Mensch mit künstlicher Intelligenz in Kontakt kommt. Darauf folgend werden im Kapitel 2  die Grundlagen der Neuronalen Netze erläutert, da diese im Umgang mit TensorFlow am häufigsten verwendet werden. Hierbei wird als erstes ein vorwärtsgerichtes Netzwerk mit seinem grundlegendem Aufbau aus verschiedenen Schichten, sowie die Berechnung der Gewichtsmatrix der dazwischenliegenden Verbindungen beschrieben. Danach erfolgt ein Überblick über die wichtigsten  Aktivierungsfunktionen wie ReLU-, Sigmoid- und Tangenshyperbolicusfunktion, außerdem wird noch auf die Kostenfunktionen eingegangen, die als Maß dienen, wie gut ein Neuronales Netz lernt. Das Kapitel 3 beschreibt das Framework TensorFlow näher, wie der Beginn der Entwicklung und die angesprochene Zielgruppe. Ein weiterer Punkt ist die Durchführung der Berechnungen mit Hilfe der Darstellung der Graphen als Datenflüsse. Hierbei leitet sich auch der Name TensorFlow, den Google dafür gewählt hat ab. Ebenso werden auch die Hard- und Software Anforderungen in diesem Kapitel beschrieben, um TF ausführen zu können und die Berechnungen auf CPU und GPU zu verteilen. Zum Abschluss wird noch der Aufbau der Architektur erklärt, sowie die Nutzung von Keras, eine speziell für Neuronale Netze geschriebene High-Level Bibliothek mit TensorFlow. Das letzte Kapitel dient der Beurteilung des Modells, zum Einen durch Aufteilung der vorhandenen Datensätze in Trainings-, Validierungs- und Testdaten und zum Anderen durch die Visualisierung mit TensorBoard, eine in TensorFlow enthaltene Webanwendung. Hierbei wird dann auf die einzelnen Visualisierungsmöglichkeiten, die sich in Skalare, Bilder, Graphen, Histogramme, Verteilungen, Projektor, Audio und Text aufteilen eingegangen, sowie die in TF nötigen Definitionen für den Programmcode.  
