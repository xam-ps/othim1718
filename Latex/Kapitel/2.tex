\chapter{Das Framework TensorFlow}

\gls{TF} ist ein \gls{ML} Framework, das f�r gro�e und variable Architekturen entwickelt wurde. Um die einzelnen Verarbeitungsschritte der Daten darzustellen, werden von \gls{TF} sogenannte Datenfluss Graphen verwendet. Diese bieten auch die M�glichkeit f�r alle Operationen festzulegen, von welcher Hardware sie berechnet werden sollen. \gls{TF} unterst�tzt dabei CPUs, GPGPUs\footnote{general-purpose graphics processing units} und eigens f�r \gls{ML} entwickelte Hardware.
%http://www.cs.virginia.edu/~gurumurthi/papers/DATE14.pdf
Besonders umfangreich unterst�tzt das Framework dabei Arbeiten im Bereich der (tiefen) \gls{NL}. Schnittstellen zu den Hochsprachen Python und C++ sollen den Einstieg erleichtern und sicherstellen, dass die verf�gbare Hardware immer bestm�glich genutzt werden kann. Mit dem sogenannte Tensorboard bringt \gls{TF} au�erdem eine Weboberfl�che mit, die ohne gro�en Aufwand f�r den Entwickler viele relevante Informationen ausgibt und teilweise auch grafisch aufbereitet.

\section{Eine Einf�hrung zu TensorFlow}

%evtl. Vergleich mit anderen Frameworks (CAFFE, Torch, ...)

\section{Der Weg des Frameworks zur Open Source Software}
%-Distbelief

%\subsection{Der Weg zur Open Source Software}



\section{Angesprochene Zielgruppe}


\section{Hard- und Software Anforderungen}



\subsection{Hardware Anforderungen}
%-CPU
%-GPU (nvidia mit CUDa support)
%-TPU
%-FPGA
\subsection{Software Anforderungen}


\section{Softwarearchitektur von TensorFlow}



